\documentclass[a4paper,11pt,dvipdfmx]{jlreq}

\usepackage{type1cm,newtxtext}
\usepackage{amsmath,amssymb, txfonts}
\usepackage{graphicx}
\usepackage{fancyhdr}
\usepackage{url}
% \usepackage{pxfonts}
\usepackage{framed}
\usepackage{url}

\setlength{\textheight}{\paperheight}
\setlength{\topmargin}{-1in}
%\addtolength{\topmargin}{-\headheight}
\addtolength{\topmargin}{-\headsep}
\addtolength{\topmargin}{20truemm}
\addtolength{\textheight}{-43truemm}
\setlength{\textwidth}{\paperwidth}
\setlength{\oddsidemargin}{-1in}
\addtolength{\oddsidemargin}{20truemm}
\setlength{\evensidemargin}{-1in}
\addtolength{\evensidemargin}{10truemm}
\addtolength{\textwidth}{-40truemm}
\setlength{\footskip}{0mm}

\newcommand{\bm}[1]{\boldsymbol{#1}}

\thispagestyle{fancy}
\lhead{セルフ原稿チェックリスト}
\rhead{提出前に必ず確認すること} % 日時を指定する場合
%\rhead{\today} % コンパイル時の日付を入れたい場合


\begin{document}

\begin{center}
\textbf{\Large
提出前に行うセルフ原稿チェックリスト %研究題目
}
\normalsize
%槇田 諭 %名前
\end{center}

\pagestyle{fancy}


\renewcommand{\labelitemi}{$\Box$}
\section{表記について}
\begin{itemize}
 \item 和文中で読点と句点は``全角''のカンマとピリオド「,.」を使用している.
 \item 誤字・脱字はもれなくチェックし,修正した.
 \item 用事・用語の漢字・かな表記は適切であることを確認した.\\
(参考例:\url{https://www.rsj.or.jp/pub/jrsj/info/words.html})
 \item 図,表,参考文献はすべて文中で参照している.
 \item 図,表,参考文献は文中の参照順に並べられている.
% \item 図は不必要に小さすぎることも大きすぎることもない.
 \item 図中に説明がある場合,十分に読み取ることのできるフォントサイズである.
 \item 図,表のキャプション(説明文)は単なる見出しにとどまらず,図,表の内容が理解できるだけの説明文を記述している.
 \item グラフを掲載する場合,縦軸と横軸の値,単位を明確に記述している.
% \item 英単語,英文の表現がある場合,カッコ書きの前に
 \end{itemize}

\section{文章表現について}
\begin{itemize}
 \item 語句の意味を辞書で調べ,適切な語句を選択・使用している.
 \item 必要以上に固有名詞を多用せず,一般名詞に置き換えて記述している\\ 
(例:ラズパイ$\rightarrow$小型コンピュータ)
 \item 固有名詞の綴りに誤りはないことを確認した.
 \item すべての文の主語,述語,目的語は明確である.
 \item すべての文の助詞は適切に使用している.
 \item 適切な接続詞を用いて,文と文のつながりが明確である.
\end{itemize}

\section{論文・報告書の内容について}

\begin{itemize}
 \item 以下の説明は過不足なく,実施した研究内容を理解するために必要な情報が記述されている.
 \item 論文,報告書を読んで生じうる疑問に対して,先んじて説明をしている.
\end{itemize}

\subsection{研究背景,先行研究,研究目的}
\begin{itemize}
 \item 序論・緒言において,研究目的が明確に示されている.
 \item 先行研究を適切に分析,参照し,自然と研究目的が導かれている.
 \item 先行研究は研究室内,国内に限らず,国際的に広く調査,分析している.
 \item 研究目的を理解するための前提となる知識は,研究背景として明瞭に記述されている.
\end{itemize}

\subsection{提案手法,実験結果}

\begin{itemize}
 \item 研究目的を達成するために取るべき手段を明瞭に記述,提案している.
 \item 説明を補足し,理解を助ける図,表を多く掲載している. 
 \item 提案手法によって得られる成果を明瞭に記述している.
 \item 提案手法の論理展開に飛躍はなく,順序立てて説明している.
 \item 実験条件,実験手順は,読者が再現するのに十分な情報が網羅されている.
 \item 掲載する実験結果に誤りはないことを確認した.
 \item 実験結果に基づいて,提案手法の有効性,誤差とその要因を検討している.
\end{itemize}


\subsection{結論,参考文献}

\begin{itemize}
 \item 結論では論文・報告書の内容をもれなく網羅し,適切にまとめられている.
 \item 結論において初出の内容は含まれず,本文中で説明されてもののみである.
 \item 参考文献は「著者名」「論文・発表題目」「論文誌名または発表会議名」「ページ番号または発表番号」「発表年」の情報がもれなく記述されている.
 \item 参考文献は読者の参考になるものを適切に参照している.(著者が参考にしたものを列挙するものではない)
\end{itemize}


\section{セルフチェックについて}
\begin{itemize}
 \item 原稿を印刷して,紙媒体で確認した.
 \item 本チェックリストを印刷して,原稿のセルフチェックを丁寧に実施した.
 \item 書き上げた文章を一晩ほど寝かせ,心身がリフレッシュした状態で上記のチェックを行った.
 \item 上記のセルフチェックを3〜5回程度は実施し,十分な修正をした.
\end{itemize}

\underline{\hspace{\columnwidth}}

\textbf{\large
$\Box$ 上記のチェック項目についてセルフチェックをもれなく丁寧に行い,原稿の完成度は十分であることを保証する.
}

\end{document}